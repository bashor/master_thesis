\documentclass{report}
%\documentclass[12pt,fleqn,a4paper]{report}

\usepackage[T2A]{fontenc}
\usepackage[english,russian]{babel}
\usepackage[utf8]{inputenc}
\usepackage{graphicx}
\usepackage{amsmath}
\usepackage{amssymb}
\usepackage{amsthm}
\usepackage{amsxtra} 
\usepackage{sty/dbl12}
\usepackage{verbatim}
\usepackage{sty/rac}
\usepackage{textcomp}
\usepackage{listings}
\usepackage{textcomp}


% \usepackage{tikz}
% \usepackage{graphics}
% \usepackage{listings}0
% \usepackage{ifthen}
% \usepackage{algorithm}
% \usepackage{booktabs}
% \usepackage[noend]{algorithmic}

%%% Z
\usepackage{float}
\usepackage{url}
\usepackage[unicode=true]{hyperref}
\usepackage{caption}
\usepackage{subcaption}

\lstdefinelanguage{JavaScript}{
	keywords={typeof, new, true, false, catch, function, return, null, catch, switch, var, if, in, while, do, else, case, break},
	% keywordstyle=\color{blue}\bfseries,
	ndkeywords={class, export, boolean, throw, implements, import, this},
	% ndkeywordstyle=\color{darkgray}\bfseries,
	% identifierstyle=\color{black},
	% sensitive=false,
	comment=[l]{//},
	morecomment=[s]{/*}{*/},
	% commentstyle=\color{purple}\ttfamily,
	% stringstyle=\color{red}\ttfamily,
	morestring=[b]',
	morestring=[b]",
	% 
	captionpos=b,
	breaklines=true,
	% basicstyle=\ttfamily,
	keywordstyle=\bfseries,
	% keywordstyle=\color{javapurple}\bfseries,
	% basicstyle=\ttfamily\color{black},
	% keywordstyle=\bfseries\color{keyword},
	morekeywords={fun, val, var, import, object, classpath, let, for, do, select, in, return, new, null, class, function},
	frame=tb
}

% \lstset{
% 	captionpos=b,
% 	breaklines=true,
% 	language=JavaScript,
% 	% basicstyle=\ttfamily,
% 	keywordstyle=\bfseries,
% 	% keywordstyle=\color{javapurple}\bfseries,
% 	% basicstyle=\ttfamily\color{black},
% 	% keywordstyle=\bfseries\color{keyword},
% 	morekeywords={fun, val, var, import, object, classpath, let, for, do, select, in, return, new, null, class, function},
% 	frame=Tb
% }

\lstnewenvironment{JavaScript}[1][]
  {\lstset{
  	language=JavaScript,
  	#1,
  }
  \noindent
  \put(0, -20){\line(1,0){470}}
  \\*\put(0, -10){\begin{Large}\path{JavaScript}\end{Large}}
  }{}

\lstnewenvironment{Kotlin}[1][]
  {\lstset{
	language=JavaScript,
	#1,
  }
  \noindent
  \put(0, -6){\line(1,0){470}}
  \\*\put(0, -5){\begin{Large}\path{Kotlin}\end{Large}}
  }{}


\renewcommand{\lstlistingname}{Листинг}

\floatstyle{plain} % optionally change the style of the new float
\newfloat{code}{H}{myc}

%\begin{code}
%\begin{lstlisting}[caption={#1}]
%    #2
%\end{lstlisting}
%\end{code}

%%%

\def\mybits{\ensuremath{\{0,1\}}}

% \tikzstyle{gate}            = [circle,fill=white,draw=black,minimum size=15pt,inner sep=1pt] % ,line width=1pt
% \tikzstyle{wire}            = [draw,->]%[draw,thick,->,auto,thick]
% \tikzset{>=latex}


% \floatname{algorithm}{Algorithm}
% \renewcommand{\algorithmicrequire}{}
% \renewcommand{\algorithmicensure}{}


\setlength{\oddsidemargin}{0.5in}

\include{header}


\binoppenalty=10000
\relpenalty=10000



% \newtheorem{conjecture}{Conjecture}[chapter]
% \theoremstyle{definition}
% \newtheorem{theorem}{Теорема}[chapter]
% \newtheorem{lemma}{Лемма}[chapter]
% \newtheorem{corollary}{Следствие}[chapter]
% \newtheorem{proposition}{Утверждение}[chapter]
% \newtheorem{fact}{Fact}[chapter]
% \newtheorem{problem}{Problem}[chapter]
% \newtheorem{exercise}{Exercise}[chapter]
% \newtheorem{example}{Пример}[chapter]
% \newtheorem{definition}{Определение}[chapter]
% \newtheorem{remark}{Замечание}[chapter]
%\newtheorem{algorithm}{Алгоритм}[chapter]
%
% \newcommand\vhdef2
% \newcommand{\hdef}[1]{{\renewcommand\vhdef2{\bfseries{#1}}\renewcommand\vhdef1}}
% \newcommand{\class}[1]{{\ifnum\vhdef=2\mathbf{#1}\else\mathrm{#1}\fi}}
% \newcommand{\co}{
% \ifnum\vhdef=2\mathbf{co\,}\else\mathrm{co\hspace{2pt}}\fi
% \ifnum\vhdef=2\textbf{-}\else\textrm{-}\fi
% }
% \newcommand{\Img}{\mathop{\mathrm{Im}}}
%
\newcommand{\lang}[1]{\mathtt{#1}}
\newcommand{\SAT}{\lang{SAT}}
\newcommand{\MAJSAT}{\lang{MAJ}\mbox{\tt-}\lang{SAT}}
\newcommand{\QBF}{\lang{QBF}}
%
\newcommand{\DSPACE}{\class{DSpace}}
\newcommand{\NSPACE}{\class{NSpace}}
\newcommand{\DTIME}{\class{DTime}}
\newcommand{\RTIME}{\class{RTime}}
\newcommand{\UTIME}{\class{UTime}}
\newcommand{\OTIME}{\class{\oplus Time}}
\newcommand{\NTIME}{\class{NTime}}
\newcommand{\BPTIME}{\class{BPTime}}
\newcommand{\BQTIME}{\class{BQTime}}
%
\renewcommand{\P}{\class{P}}
\newcommand{\Ppoly}{\class{P}/\class{poly}}
\newcommand{\LOG}{\class{LOG}}
\newcommand{\ZPP}{\class{ZPP}}
\newcommand{\RP}{\class{RP}}
\newcommand{\coRP}{\class{{\co}RP}}
\newcommand{\UP}{\class{UP}}
\newcommand{\coUP}{\class{{\co}UP}}
\newcommand{\FewP}{\class{FewP}}
\newcommand{\coFewP}{\class{{\co}FewP}}
\newcommand{\Few}{\class{Few}}
%
\newcommand{\NP}{\class{NP}}
\newcommand{\coNP}{\class{{\co}NP}}
\newcommand{\SigmaP}[1]{\Sigma^{#1}\class{P}}
\newcommand{\NC}{\class{NC}}
%
\newcommand{\PiP}[1]{\Pi^{#1}\class{P}}
\newcommand{\DeltaP}[1]{\Delta^{#1}\class{P}}
\newcommand{\BPP}{\class{BPP}}
\newcommand{\EQP}{\class{EQP}}
\newcommand{\RQP}{\class{RQP}}
\newcommand{\BQP}{\class{BQP}}
\newcommand{\PH}{\class{PH}}
\newcommand{\PCP}{\class{PCP}}
%
\newcommand{\OP}{\class{\oplus P}}
\newcommand{\AWPP}{\class{AWPP}}
\newcommand{\PP}{\class{PP}}
\newcommand{\IP}{\class{IP}}
\newcommand{\PSPACE}{\class{PSPACE}}
%
\newcommand{\EXP}{\class{EXP}}
\newcommand{\MIP}{\class{MIP}}
\newcommand{\NEXP}{\class{NEXP}}
\newcommand{\coNEXP}{\class{{\co}NEXP}}
%
\newcommand{\BPO}{\class{BP}\cdot}
\newcommand{\OPO}{\class{OP}\cdot}
%
\newcommand{\PEXP}{\class{PEXP}}
\newcommand{\EXPSPACE}{\class{EXPSPACE}}
\newcommand{\MAEXP}{\class{MA}_\class{EXP}}
%
\newcommand{\vs}{=\!\!\!?\!\!\!\!=}
\newcommand{\Retc}{Й\;Ф.\;Д.}
\newcommand{\poly}{\mathrm{poly}}
%
\newcommand{\substd}{16d\log^2_*n}
\newcommand{\substf}{17d\log_*^2n}
%{33d'\log_*^2n}
\newcommand{\substk}{12 d' \ln (16 d\log_*^2n)} % could be 8
%{12d\ln(64d'\log_*^2n)}  %or d'?
\newcommand{\substl}{(3.03\cdot 10^4)\cdot f\cdot\ln (16kd\log_*^2n)} % could be 2
%2\cdot10^4\cdot f\ln(64d'k\log_*^2n)} %or l=8..?
%
\newcommand{\errfhigh}{\frac1{0.99f}}
\newcommand{\errhighsubst}{\frac1{16d\log_*^2n}} %CO% 17*0.99
\newcommand{\errmiddle}{\frac1{\errf}}
\newcommand{\inverrlowsubst}{18d\log_*^2n}
\newcommand{\errlowsubst}{\frac1{\inverrlowsubst}}
\newcommand{\testerrflowproved}{\frac1{1.01f}}
\newcommand{\testerrflowused}{\frac1{1.011f}}
%
\newcommand{\seperrfhigh}{\frac1{0.99f}}
\newcommand{\seperrhighsubst}{\frac1{16d\log_*^2n}} %CO% 17*0.99
\newcommand{\seperrmiddle}{\frac1{\errf}}
\newcommand{\sepinverrlowsubst}{18d\log_*^2n}
\newcommand{\seperrlowsubst}{\frac1{\inverrlowsubst}}
\newcommand{\septesterrflowproved}{\frac1{1.01f}}
\newcommand{\septesterrflowused}{\frac1{1.011f}}
%
\newcommand{\testerrrej}{e^{-\frac{l}{3.03\cdot 10^4\cdot f}}}
\newcommand{\testerracc}{e^{-\frac{l}{2\cdot 10^4\cdot f}}}
%
\newcommand{\septesterrrej}{e^{-\frac{l}{3.03\cdot 10^4\cdot f}}}
\newcommand{\septesterracc}{e^{-\frac{l}{2\cdot 10^4\cdot f}}}
%
\newcommand{\certerrrej}{e^{-\frac{k}{12d'}} + k\cdot \testerrrej}
\newcommand{\certerracc}{e^{-\frac{k}{8d'}}  + k\cdot \testerracc}
%
\newcommand{\sepcerterrrej}{2e^{-\frac{k}{12d'}} + 2k\cdot \septesterrrej}
\newcommand{\sepcerterracc}{2e^{-\frac{k}{8d'}}  + 2k\cdot \septesterracc}
%
\newcommand{\certerrrejsubst}{\frac1{8 d\log_*^2n}} 
\newcommand{\certerraccsubst}{\frac1{8 d\log_*^2n}} %CO% 
%
\newcommand{\sepcerterrrejsubst}{\frac1{8 d\log_*^2n}} 
\newcommand{\sepcerterraccsubst}{\frac1{8 d\log_*^2n}} %CO% 
%
\newcommand{\hps} {heuristic proof system\xspace}
\newcommand{\hpss}{heuristic proof systems\xspace}
\newcommand{\ahps}{automatizable \hps}
\newcommand{\Nat}{\mathbb{N}}
%
\newcommand{\ournote}[1]{\ref{?}\footnote{\selectlanguage{russian}#1}}
\newcommand{\ourcomment}[1]{\par\fbox{\begin{minipage}{14cm}\small
	{\bfseries Calculations (to be removed):}\\ #1\end{minipage}}\par}

\DeclareMathOperator*{\supp}{supp}
\DeclareMathOperator*{\Exp}{E}
\DeclareMathOperator*{\rk}{rk}

\newcommand{\doubletext}[2]{
	\noindent\begin{minipage}[b]{0.5\linewidth}
        #1
	\end{minipage}
	\noindent\begin{minipage}[b]{0.45\linewidth}
        \begin{flushright}
            #2
        \end{flushright}
    \end{minipage}
}

%Ivan Bliznets new commands
\newcommand{\maxsat}{MAX-SAT }
\newcommand{\nthreemaxsat}{$(n,3)$-MAX-SAT }
\newcommand{\nne}[1]{\ensuremath{\bar{#1}}}
\newcommand{\SRf}[1]{\textrm{SR}\ensuremath{$5${#1}}}
\newcommand{\SRs}[1]{\textrm{BR}\ensuremath{$2${#1}}}
\newcommand{\superR}{\textbf{"SuperResolution"}}





\newcommand{\algname}[1]{\textrm{\sc #1}}
\newcommand{\mainalgname}{\algname{MAX-SAT-Alg}}
\newcommand{\nthreealgname}{\algname{$(n,3)$-MAX-SAT-Alg}}
\newcommand{\singletonalgname}{\algname{Singleton-MAX-SAT-Alg}}
\newcommand{\mscalgname}{\algname{msc}}

%\newcommand{\simple}[1]{Правило упрощения SR{#1}}

\newtheorem{simple}{Правило упрощения}
\newtheorem{branch}{Правило расщепления}
\newtheorem{rusremark}{Замечание}


\newcommand{\nMax}{$(n,3)$-MAX-SAT }
\newcommand{\alg}{\textrm{N3MaxSat}}


\newcommand{\rul}[2]{\textbf{#2#1}}


\newcommand{\myif}{\textbf{If }}
\newcommand{\myth}{\textbf{then }}
\newcommand{\myreturn}{\textbf{return}}

\newcommand{\mytrueclause}{({\cal T})}
