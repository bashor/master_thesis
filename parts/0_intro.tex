\startprefacepage

%\section { Введение.}

% Здесь обычно обосновываются актуальность выбранной темы, цель и содержание поставленных задач, формулируются объект и предмет исследования, указывается избранный метод (или методы) исследования, сообщается, в чем заключается теоретическая значимость и прикладная ценность ожидаемых результатов.

% Актуальность – обязательное требование к любой работе. В применении к квалификационной работе важно, как ее автор умеет выбрать тему и насколько правильно он эту тему понимает и оценивает с точки зрения научной и социальной значимости, т. к. это характеризует его профессиональную подготовленность. 

% Обоснование актуальности не должно быть многословным, после него логично перейти к формулировке цели предпринимаемого исследования, а также указать на конкретные задачи, которые предстоит решать в соответствии с этой целью. Это обычно делается в форме перечисления (изучить., описать., установить., выявить., вывести формулу… и т.п.)

% Обязательным элементом введения является также указание на методы исследования, которые служат инструментом в добывании фактического материала, являясь необходимым условием достижения поставленной в работе цели. 
% В конце вводной части желательно раскрыть структуру работы, то есть дать перечень ее структурных элементов и обосновать последовательность их расположения.


% (изучить., описать., установить., выявить., вывести формулу… и т.п.)

Последние 5-10 лет на фоне бурного развития интернета JavaScript получила широкое распросторонение. Но по ряду причин JavaScript не устраивает разработчиков и начали появляться трансляторы в JavaScript из других языков, кроме того это дает возможность разработчикам писть и серверный, и клиентсыки код на одном языке.

С 2010 года в комапинии JetBrains разрабатывается язык программирования Kotlin. В 2012 году появилась возможность компилировать код на Kotlin в JavaScript. Компилятор выдавал код удовлетворительного качества, но хотелсь большего. Тем более, конкуренты/аналоги, по состоянию на начало 2013 года ушили далеко вперед.

Таким образом в рамках данной работы хочется улучшить качество генерируемого компилятором Kotlin в JavaScript. В первую очередь, хочется улучшить производительность получаемого JavaScript-кода.

Выявить узкие <<места>> и предложить способы их устранения.



JavaScript -- это язык програмимрования для Веб. Подовляющее большинство веб-сайтов используют JavaScript, и все современные веб-браузеры -- для настольных компьютеров, игровые приствок, эклектронных планшетов и смартфонов включают интерпритатор JavaScript, что делает JavaScript самым широкоприменимым языком программирования из когда либо существовавших в истории. JavaScript входит в тройку техналогий, которые должны знать  любой веб-разработчик: язык разметки HTML, позволяющий определить содержимов веб-страниц, язык стилей CSS, позволяющий определить внешний вид веб-страниц, и язык программирования JavaScript, позволяющий определять поведение веб-страниц.

JavaScript является высокоуровневым, динамическим, нетипизрованым и интрепретируемым языком программирования, который хорошо подходит для программирования в объектно-ориентированном и функциональном стилях. Свой синтаксис JavaScript унаследовал из языка Java, свои первоклассные функции – из языка Scheme, а механизм наследования на основе прототипов – из языка Self. 

Название языка «JavaScript» может вводить в заблуждение. За иснлюкечинем поверхностной синтаксической схожести, JavaScript полностью отличается от языка программирования Java. JavaScript давно перерос рамки языка сценариев, превратившись в надежный и эффективный универсальный язык программирования. Последняя версия языка (смотрите врезку) определяет множество новых особенностей, позволяющих использовать его для разработки крупномасштабного программного обеспечения.


Чтобы представлять хоть какой-то интерес, каждый язык программирования должен иметь свою платформу, или стандартную библиотеку или API функций для выполнения таких базовых операций, как ввод и  вывод. Ядро языка JavaScript определяет минимальный прикладной интерфейс для работы с текстом, массивами, датами и регулярными выражениями, но в нем отсутствуют операции ввода-вывода. Ввод и вывод (а также более сложные возможности, такие как сетевые взаимодействия, сохранение данных и работа с графикой) перекладываются на «окружающую среду», куда встраивается JavaScript. Обычно роль окружающей среды играет веб-броузер (однако в главе 12 мы увидим два примера использования JavaScript без привлечения веб-броузера).

JavaScript – объектно-ориентированный язык, но используемая в нем объектная модель в корне отличается от модели, используемой в большинстве других языков. 


---

JavaScript: названия и версии
JavaScript был создан в компании Netscape на заре зарождения Веб. Название «JavaScript» является торговой маркой, зарегистрированной компанией Sun Microsystems (ныне Oracle), и  используется для обозначения реализации языка, созданной компанией Netscape (ныне Mozilla). Компания Netscape представила язык для стандартизации европейской ассоциации производителей компьютеров ECMA (European Computer Manufacturer’s Association), но из-за юридических проблем с торговыми марками стандартизованная версия языка получила несколько неуклюжее название «ECMAScript». Из-за тех же юридических проблем версия языка от компании Microsoft получила официальное название «JScript». Однако на практике все эти реализации обычно называют JavaScript. В течение прошлого десятилетия все веб-броузеры предоставляли реализацию версии 3 стандарта ECMAScript, и в действительности разработчикам не было необходимости задумываться о номерах версий: стандарт языка был стабилен, а его реализации в веб-броузерах в значительной мере были совместимыми. Недавно(???) вышла новая важная версия стандарта языка под названием ECMAScript 5, и к моменту написания этих строк производители броузеров приступили к созданию его реализации. Эта книга охватывает все нововведения, появившиеся в ECMAScript 5, а также все особенно
сти, предусмотренные стандартом ECMAScript 3. (Четвертая версия стандарта ECMAScript разрабатывалась много лет, но из-за слишком амбициозных целей так и не была выпущена.) Однако иногда можно встретить упоминание о версии JavaScript, например: JavaScript 1.5 или JavaScript 1.8. Эти номера версий присваивались реализациям JavaScript, выпускаемым компанией Mozilla, причем версия 1.5 соответствует базовому стандарту ECMAScript 3, а более высокие версии включают нестандартные расшире
ния. Наконец, номера версий также присваиваются отдельным интерпретаторам, или «механизмам» JavaScript. Например, компания Google разрабатывает свой интерпретатор JavaScript под названием V8, и к моменту написания этих строк текущей версией механизма V8 была версия 3.0.\cite{JsDef6}


---
Небольшое введение:
JavaScript — прототипно-ориентированный скриптовый язык программирования с динамической типизацией. Является диалектом языка ECMAScript.
Kotlin — статически типизированный объектно-ориентированный язык, разрабатываемый в компании JetBrains. Компилируется в Java байт-код и в JavaScript.

---

JavaScript is the x86 of the web
Brendan Eich (создатель JavaScript)

...JavaScript is the VM of the web. ....
Douglas Crockford (известный JS-разработчик, создатель JSON)

JavaScript is an assembly language for the Web. ...
Erik Meijer (language designer из Microsoft,
<<отец>> LINQ )

Почему JS? JS за последнее время стал очень популярен и его часто называют ассемблером для веба. Сейчас наверно почти для любого языка есть какой нибудь транслятор/компилятор в JS. 

---
Почему JavaScript?

Интерпретатор JS есть почти везде
JS не очень удобен для разработки больших проектов
Удобно писать все части системы на одном языке



Почему так происходит и почему команде котолина нужно изобретать свой ввелосипед? Я вижу несколько основных причин, думаю, общих для всех подобных проектов:
1. JS есть почти везде (потому что там есть браузер)
2. Но JS не очень удобен для разработки больших и сложных проектов (да и вообще для разработки)
3. Удобно писать и серверную и клиентскую часть(если такие есть) на одном языке. Да можно писать серверную часть на JS, но смотри 2
Дальше мнения могут расходится, например кому то нравится статическая типизация, кому то динамическая у каждого подхода есть свои минусы и плюсы...
или кому то нравятся много скобочек
Kotlin новый языка который тоже нуждается в таком трансляторе.
Ввиду специфики языка не так просто взять и переиспользовать готовые решения. Готовые решения обычно заточены под свой конкретный язык
Сейчас уже есть какая то реализация данного транслятора, но он еще достаточно далек от того что хотелось бы иметь.

---
Цель
Улучшить компилятор Kotlin в JavaScript так, чтобы он генерировал более эффективный в плане производительности код.

---
Задачи
Изучить аналоги
Изучить существующие JS виртуальные машины
Выбрать и реализовать тесты производительности для сравнения (бенчмарки)
Проанализировать результаты тестов и выделить <<узкие места>>
Предложить способы их оптимизации
Реализовать предложенные решения

Для достижения данной цели были поставлены следующие задачи
Изучить аналоги
Изучить существующие JS виртуальные машины (они давно перестали быть интерпритаторами)
Выбрать и реализовать тесты производительности для сравнения (бенчмарки)
Проанализировать результаты тестов и выделить <<узкие места>>
Предложить способы их оптимизации
Реализовать предложенные решения


Аналоги (1/2)
Dart — язык программирования, созданный Google, позиционируется в качестве замены JavaScript
Dart2js — компилятор Dart в JavaScript. Разрабатывается той же командой что и Dart
CoffeeScript — иной взгляд на JavaScript. Многое взято из Ruby, Python, Haskell и Erlang
TypeScript — язык от Microsoft, позиционируемый как средство разработки веб-приложений

Ceylon — язык, разрабатываемый в RedHat. Компилируется в Java байт-код и JavaScript
Google Web Toolkit — Java-фреймворк для создания веб-приложений. Включает в себя компилятор из Java в JavaScript
Closure-Compiler — оптимизатор JavaScript - кода
Pyjs, qb.js, ghcjs, ClojureScript, ...


Большой обзор на altJs.org

Ceylon во многом похожий на Котлин проект, но с ними не интересно сравниваться, потому что они не сильно заморачиваются на оптимальности генерируемого JS кода

GWT
Closure-Compiler

Отдельно хочется отметить GWT и CC 

Все это проекты я изучил и перенял опыт.

---
JS виртуальные машины

 
Разработчик
Где используется?
v8
Google
Chrome, Node.js,
Android, ...

SpiderMonkey
Mozilla 
Firefox,
Thunderbird, ...

Chakra
Microsoft
IE ver. $\geq$ 9

Rhino
Mozilla
 
JavaScriptCore
WebKit
Safari, iOS ...

---
С кем сравниваемся?

JavaScript (бенчмарки от разработчиков v8)
Dart (бенчмарки от разработчиков)
Dart2js


Есть набор тестов которые используются практически всеми разработчиками JS движков. И было бы логично взять эти тесты и для моей работы, потому что можно взять использовать уже готовые, написанные специалистами тесты, тем самым исключив предвзятость с моей стороны. И конечно же я тоже опубликую свои тесты и каждый сможет сам провести такие же эксперименты как и я.

Отдельно хочется отметить последний тест -- этот бенчмарк пока еще не является общепринятым, но все же имеется реализация для многих языков. Данный дест был разработан разработчиками их гугла в 2011 с целью сравнения C++, Java, Scala, Go по таким критериями как производительность, объем кода, обхем бинарника, объем используемой памяти. 
Данный бенчмарк я также переписал на котлин, но пока не исследовал.

Тесты отмеченные одной звездочкой не реализованы.

Richards
OS kernel simulation benchmark, originally written in BCPL by Martin Richards (539 lines).
Main focus: property load/store, function/method calls
Secondary focus: code optimization, elimination of redundant code

Deltablue
One-way constraint solver, originally written in Smalltalk by John Maloney and Mario Wolczko (880 lines).
Main focus: polymorphism
Secondary focus: OO-style programming
---
Сравниваться со всем кончено получилось бы и пришлось выбрать нескольких.
Почему именно эти аналоги?
Ну во первых это реализация на чистом JS, были взяты тесты от разработчиков v8. Я считаю это обязательным пунктом, чтобы показать, понять на сколько мы хуже чем код написанный вручную на целевом языке.(тем более экспертами) И сделать выводы в чем мы хуже и почему, что можно улучшить. А в итого показать на сколько мы хуже или лучше чем такой код.

Казалось бы что может быть лучше вручную написанного JS кода? Я решил немного повысить планку и взять Dart, особенность в том, что дарт спроектирован с учетом некоторых недостатков JS, язык разрабоатывается практически той же коммандой которая разрабатывает v8, нативная реализация дарта выполняется в своей виртуальной машине которая знает о коде чуть больше и очень активно пользуется этим, за счет чего на многих тестах уже сейчас показывает лучшие результаты чем вручную написанный JS

Ну и планачка чуть по ниже это js полученный путем трансляции дарт в js

ТО план минимум работать так же хорошо как dart2js, план максимум работать так же хорошо как dart(видимо не всегда достижимый)
---
Richards
Данный бенчмар симулирует работу ядра ОС, написан в конце 90х Мартином Ричардсом.
Основной фокс теста -- это чтение и запись данных и вызов функций- методов

OS kernel simulation benchmark, originally written in BCPL by Martin Richards (539 lines).
Main focus: property load/store, function/method calls
Secondary focus: code optimization, elimination of redundant code

---
Richards: Выводы

Оптимизировать доступ к полям/свойствам, по возможности, сохранив бинарную совместимость
Сделать аналог импорта, предоставив тем самым возможность кратко обращаться к объектам и их полям
Необходимо упростить структуру генерируемого кода

---
Deltablue
constraint solver написанный John Maloney and Mario Wolczko в конце 80х, начале 90х (?)
Основной фокус данного теста -- полиморфизм, ООП стиль программирования. В тесте очень часто и в большем количестве создаются объекты. Эти объекты живут не очень долго и видимо этот тест так же подходит для тестирования GC, но это нас не сильно интересует

One-way constraint solver, originally written in Smalltalk by John Maloney and Mario Wolczko (880 lines).
Main focus: polymorphism
Secondary focus: OO-style programming

---
<<Прямой>> вызов конструкторов
---
DeltaBlue: Выводы

Заменить все контейнеры на родные для JS аналоги
Генерировать чистое прототипное наследование без оберток
Инициализировать все поля класса в конструкторе
Инициализировать все поля класса в одном порядке
Заменить или оптимизировать Kotllin.equals

---
Что дальше?

Реализовать предложенные оптимизации
Написать новые бенчмарки

Если новые бенчмарки не дадут результатов: (или если будет время)
Inlining функций и функциональных литералов
Оптимизации на основе  Continuation-passing style и 
Static single assignment
