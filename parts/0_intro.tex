\startprefacepage

%\section { Введение.}

JavaScript сегодня является самым популярным языком программирования, используемым для разработки веб-приложений на стороне клиента\cite{JsUsage, LangUsage}. В статье <<Самый неправильно понятый язык программирования в мире стал самым популярным в мире языком программирования>>\cite{MostPopLang} Дуглас Крокфорд утверждает, что лидирующую позицию JavaScript занял в связи с развитием AJAX, поскольку браузер стал превалирующей системой доставки приложений. Он также констатирует растущую популярность JavaScript, то, что этот язык встраивается в приложения, отмечает значимость языка.

Широкому распространению языка Javascript так же способствовал тот факт, что он является единственным языком программирования, который поддерживается всеми современными браузерами. Таким образом, чтобы разрабатывать код для клиентской части программы на другом языке, отличном от JavaScript необходимо уметь компилировать код на данном языке  в JavaScript. Такая задача часто возникает из-за ряда недостатков языка JavaScript, затрудняющих разработку больших приложений.
%todo заменить на http://www.2ality.com/2013/04/12quirks.html 
Например, Axel Rauschmayer в серии статей описывает странности присущие языку JavaScript\cite{JsQuirks}, рассмотрим лишь некоторые из них:
\begin{itemize}
\item Неявная конвертация значений переменных.
\item Две константы \path{undefined} и \path{null} имеющую похожую семантику -- отсутствие какого-либо значения. 
\item Запутанная семантика операторов сложения и сравнения.
\item Запись в неопределенную переменную приводит к созданию глобальной переменной.
\item Привязка областей видимости переменных к функциям, а не блокам и <<всплытие>> определений переменных и функций.
\item Реализация замыканий -- когда замыкается все лексическое окружение объемлющей функции.
\end{itemize}
%
% К примеру, Дуглас Крокфорд в своей книге <<JavaScript: The Good Parts>>\cite{GoodParts} указывает более 20 неудачных решений, сделанных при проектировании языка. Среди~которых:
% \begin{itemize}
% \item Использование глобальных переменных в стандартных API.
% \item Привязка областей видимости переменных к функциям\\* (а не блокам).
% \item Запутанная семантика операторов сложения и сравнения.
% \item Наличие неиспользуемых ключевых слов.
% \end{itemize} 

Кроме того, динамическая типизация языка затрудняет разработку инструментов для JavaScript таких, как интегрированные среды разработки (IDE) и статические анализаторы кода. Все эти обстоятельства, а также растущая сложность клиентских приложений, которые приходится реализовывать разработчикам, подталкивают программистов к поиску замены для JavaScript. Благодаря этому возникло множество программных продуктов, осуществляющих трансляцию в JavaScript многих современных языков программирования.\cite{AltJS}

С 2010 года в компании JetBrains разрабатывается язык программирования Kotlin. А в 2012 году данный язык обзавелся компилятором в JavaScript. Высокая конкуренция между язками стремящимися занять место языка JavaScript, как основного языка разработки клиентского кода, способствовала их быстрому развитию. В связи с чем стала актуальной задача улучшения качества кода генерируемого компилятором Kotlin в JavaScript.

В данной работе проводится анализ качества кода, генерируемого компилятором Kotlin в JavaScript, с целью выявить и устранить в нем <<узкие>> места. %Целью работы является улучшить компилятор Kotlin в JavaScript так, чтобы он генерировал более эффективный в плане производительности код.

% Здесь обычно обосновываются актуальность выбранной темы, цель и содержание поставленных задач, формулируются объект и предмет исследования, указывается избранный метод (или методы) исследования, сообщается, в чем заключается теоретическая значимость и прикладная ценность ожидаемых результатов.

% Актуальность – обязательное требование к любой работе. В применении к квалификационной работе важно, как ее автор умеет выбрать тему и насколько правильно он эту тему понимает и оценивает с точки зрения научной и социальной значимости, т. к. это характеризует его профессиональную подготовленность. 

% Обоснование актуальности не должно быть многословным, после него логично перейти к формулировке цели предпринимаемого исследования, а также указать на конкретные задачи, которые предстоит решать в соответствии с этой целью. Это обычно делается в форме перечисления (изучить., описать., установить., выявить., вывести формулу… и т.п.)

% Обязательным элементом введения является также указание на методы исследования, которые служат инструментом в добывании фактического материала, являясь необходимым условием достижения поставленной в работе цели. 
% В конце вводной части желательно раскрыть структуру работы, то есть дать перечень ее структурных элементов и обосновать последовательность их расположения.

% (изучить., описать., установить., выявить., вывести формулу… и т.п.)