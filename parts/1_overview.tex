\chapter{Обзор}

% Выпускная работа может и должна эффективно контролировать не только практические, расчетные навыки, но и общий уровень фундаментальной образованности выпускника. Самостоятельное выполнение и защита выпускной работы позволяет выявить творческие способности, инициативность, научную самостоятельность студента и дать лучшим выпускникам обоснованную рекомендацию на продолжение образования в аспирантуре.
% В выпускной работе автор должен подвести итоги проведенным на сегодняшний день исследованиям (своим и/или чужим), а также обосновать план проведения дальнейших работ.
% Выпускная работа может составлять начальную ступень большого исследования, которое должно вылиться в кандидатскую диссертацию, но представляет собой законченную разработку. Выполненная таким образом выпускная работа позволит проконтролировать творческие способности, степень понимания современных научно-технических проблем; умение использовать методы научного исследования, уровень владения математическим моделированием физических процессов, знания и умения в области физики, математики, информатики, и пр.


В данном разделе приводятся результаты анализа существующих решений смежных
задач, а также дается обоснование актуальности поставленной задачи.

\section{Язык программирования JavaScript}

В настоящее время JavaScript, разработанный в 1995 году, является наиболее
популярным из языков для разработки клиентских скриптов для веб приложений, а стандарт
ECMAScript 5.0 [6] поддерживается всеми современными браузерами. Также в последнее
время JavaScript набирает популярность в качестве языка для разработки серверной части
приложений.
Однако, несмотря на свою распространенность, язык JavaScript имеет существенное
количество недостатков, затрудняющих разработку больших приложений. К примеру,
Douglas Crockford
в своей книге “JavaScript: The Good Parts” [7] указывает более 20
неудачных решений, сделанных при проектировании языка. Среди них:
- Использование глобальных переменных в стандартных API.
- Привязка областей видимости переменных к функциям (а не блокам).
- Запутанная семантика операторов сложения и сравнения.
- Наличие неиспользуемых ключевых слов.
Кроме того, динамическая типизация языка затрудняет разработку инструментов для
JavaScript таких, как интегрированные среды разработки (IDE) и статические анализаторы
кода.
Все эти обстоятельства, а также растущая сложность клиентских приложений,
которые приходится реализовывать разработчикам, подталкивают программистов к поиску
замены для JavaScript. Благодаря этому возникло множество[8] программных продуктов,
осуществляющих трансляцию в JavaScript многих современных языков программирования.
6

\section{Существующие трансляторы в JavaScript}

Автором был проведен анализ уже существующих решений с целью выявления
актуальности поставленной задачи и изучения возможности повторного использования
компонентов готовых продуктов. Далее приводится краткая техническая характеристика
наиболее популярных из них:

\subsection{Google Web Toolkit Java-to-JavaScript Compiler}

Google Web Toolkit Java-to-JavaScript Compiler является частью каркаса Google Web
Toolkit (GWT)[9], который был анонсирован компанией Google в 2006 году, текущая версия
2.4.0. выпущена в сентябре 2011 года. Задачей данного модуля является трансляция
программ на языке Java в кросс-браузерные скрипты JavaScript.
К преимуществам разработки под каркас Google Web Toolkit по сравнению с
JavaScript можно отнести:
- Простоту в освоении для Java – программистов.
- Отличную поддержку в современных IDE языка Java и каркаса GWT.
- Возможность использовать значительную часть стандартных библиотек Java.
- Богатую библиотеку элементов интерфейса.
- Оптимизирующий генератор кода.
- Возможность отладки прямо в браузере.
К недостаткам можно отнести:
- Невысокую скорость генерации JavaScript.
- Сложный механизм взаимодействия с библиотеками, написанными на
JavaScript.
-
Невозможность безболезненно отказаться от использования GWT (перейти к
использованию другой технологии без переписывания значительной части
кода).

\subsection{Язык программирования Dart}

Язык программирования Dart[10], который был аносирован компанией Google в
сентябре 2011 года, является попыткой введения нового стандарта в качестве языка
7
разработки клиентских скриптов для браузеров (вместо JavaScript). Dart поддерживает
объектно-ориентированную и функциональные парадигмы и использует смесь статической и
динамической типизации. Он может быть исполнен на виртуальной машине DartVM
браузера Dartium, а модуль dartc осуществляет трансляцию Dart в JavaScript для исполнения
в других браузерах.
К преимуществам разработки на языке Dart можно отнести:
-
Современный выразительный язык, спроектированный специально для
разработки клиентских скриптов.
-
Обширную стандартную библиотеку.
Стоит отметить, что сам язык и инструменты для работы с ним на данный момент
находятся в разработке.

\subsection{Язык программирования CoffeeScript}

CoffeeScript[11] – язык, призванный облегчить жизнь разработчикам на JavaScript.
Транслятор CoffeeScript осуществляет перевод CoffeeScript в JavaScript для исполнения в
браузере. За небольшое время своего существования CoffeeScript (версия 0.1 выпущена в
конце 2009 года, в мае 2012 года выпущена версия 1.3) приобрел популярность, благодаря
компактному и
выразительному синтаксису,
а
также
легкости
в
изучении
для
программистов, знакомых с JavaScript.
К преимуществам разработки на языке CoffeeScript по сравнению с JavaScript можно
отнести:
- Выразительный синтаксис.
- Поддержку классов.

\subsection{Язык программирования Haxe}

Язык программирования Haxe[12] появился в 2005 году в качестве замены языка
ActionScript. Haxe поддерживает объектно-ориентированную и функциональные парадигмы
и имеет статическую типизацию с поддержкой динамических типов. Основной идеей языка
Haxe является использование одного языка для разработки под множество платформ. На
8
июнь 2012 года Haxe может быть транслирован в код для следующих платформ: JavaScript,
C++, Flash, PHP 5, NekoVM.
К преимуществам разработки на языке Haxe можно отнести:
-
Возможность разработки на одном языке программирования под множество
платформ.
-
Кросс-платформенную стандартную библиотеку.

\section{Актуальность задачи}

Ни одно из рассмотренных автором решений не удовлетворяет следующим
критериям:
-
Статически типизированный язык программирования с компактным и
выразительным синтаксисом.
- Существующий компилятор для платформы JVM.
- Хорошая поддержка в интегрированной среде разработки.
Поэтому является обоснованным решение реализовывать новый программный
продукт. В качестве языка программирования, удовлетворяющим заданным критериям был
выбран язык Kotlin.

\section{Язык программирования Kotlin}

Kotlin[5] – один из нескольких новых развивающихся языков программирования для
платформы
JVM.
Он
имеет
статическую
типизацию
и
поддерживает
объектно-
ориентированную и функциональную парадигмы. Kotlin был анонсирован компанией
JetBrains в июле 2011года, а в июне 2012 года была выпущена версия M2. Отличительной
особенностью Kotlin является его полная совместимость с языком Java.
Проведем
сравнение
нашего
решения
с
уже
существующими
продуктами,
упомянутыми в 2.2.
Согласно[13], к преимуществам разработки на языке Kotlin по сравнению с Java и
соответственно каркасом GWT можно отнести:
-
Более лаконичный и выразительный синтаксис.
9
-
Возможность обнаружить большее количество проблем на стадии компиляции
(более безопасная система типов).
- Более полная поддержка множественного наследования.
- Поддержка функций высших порядков и замыканий.
- Локальный вывод типов.
По сравнению с языками Dart и CoffeeScript основным преимуществом Kotlin
является статическая типизация, которая в частности означает возможность более полной
поддержки в интегрированных средах разработки.
Из перечисленных выше языков наиболее близким к Kotlin языком является Haxe,
однако Haxe за более чем 5 лет своего существования так и
не получил большого
распространения за исключением разработки для платформы Flash. (Автор предполагает, что
одними из причин непопулярности Haxe являются отсутствие поддержки хотя бы одной из
наиболее популярных платформ, CLR (Common Language Runtime[14]) или JVM, и
отсутствие качественной поддержки в интегрированных средах разработки).
