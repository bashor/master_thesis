\startreferatpage

{\Large

В результате выполнения данной работы были реализованы тесты оптимизации, с помощью которых были выявлены <<узкие>> места в текущей реализации компилятора Kotlin в JavaScript. Кроме того, были предложены оптимизации для устранения выявленных недостатков, часть из которых были реализованы и используется в официальной версии компилятора Kotlin.

Работа выполнена на $41$ страницах в $3$ главах, использовано $42$ источника.

% Пояснительная записка содержит 48 страниц, включая 7 таблиц и 27 листингов с примерами исходных кодов.


% Реферат должен в кратком виде, в объеме до одной страницы, содержать цель и объект исследования, полученные новые результаты, область их возможного применения, а также данные об объеме работы, количестве разделов, иллюстраций, таблиц, приложений, использованных источников.
% Текст реферата содержит изложение основных результатов работы и сделанных на их основе выводов. Основная задача реферата -- передать максимум информации в тексте минимального объема при сохранении такой формы изложения, которая способствует быстрому восприятию содержащихся в нем сведений.

Работа состоит из трех глав, введения и заключения.
Во введении в краткой форме объясняются причины возникновения необходимости выполнения такого рода работы.
В первой главе описываются цели и задачи данной работы.
Во второй главе производится краткий обзор предметной области, обосновываются выбор тестов и языков для исследования.
Третья, основная глава, посвящена анализу результатов тестов и реализации оптимизаций.
В заключении подводятся итоги и описываются возможные направления дальнейших работ.