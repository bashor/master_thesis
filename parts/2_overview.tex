\chapter{Обзор предметной области}

\section{Язык программирования JavaScript}

JavaScript -- это язык программирования для веб. Подавляющее большинство веб-сайтов используют JavaScript, и все современные веб-браузеры -- для настольных компьютеров, игровых приставок, электронных планшетов и смартфонов, включают интерпретатор JavaScript, что делает JavaScript самым широкоприменимым языком программирования из когда либо существовавших в истории \cite{JsDef6}. JavaScript входит в тройку технологий, которые должны знать  любой веб-разработчик: язык разметки HTML, позволяющий определить содержимое веб-страниц, язык стилей CSS, позволяющий определить внешний вид веб-страниц, и язык программирования JavaScript, позволяющий определять поведение веб-страниц.

JavaScript является высокоуровневым, динамическим и интерпретируемым языком программирования, который хорошо подходит для программирования в объектно-ориентированном и функциональном стилях. Многое в своем синтаксисе JavaScript унаследовал из языка Java, возможность использовать функции как значения -- из языка Scheme, а механизм наследования на основе прототипов -- из языка Self\cite{JsDef6}. 

%Название языка «JavaScript» может вводить в заблуждение. За исключением поверхностной синтаксической схожести, JavaScript совершенно не похож на язык программирования Java. JavaScript давно перерос рамки языка сценариев, превратившись в надежный, эффективный и универсальный язык программирования. Последняя версия языка определяет множество новых особенностей, позволяющих использовать его для разработки крупномасштабного программного обеспечения\cite{JsDef6}.
% Чтобы представлять хоть какой-то интерес, каждый язык программирования должен иметь свою платформу, или стандартную библиотеку или API функций для выполнения таких базовых операций, как ввод и  вывод. Ядро языка JavaScript определяет минимальный прикладной интерфейс для работы с текстом, массивами, датами и регулярными выражениями, но в нем отсутствуют операции ввода-вывода. Ввод и вывод (а также более сложные возможности, такие как сетевые взаимодействия, сохранение данных и работа с графикой) перекладываются на «окружающую среду», куда встраивается JavaScript. Обычно роль окружающей среды играет веб-броузер.


JavaScript является объектно-ориентированным языком, но используемое в языке <<прототипное наследование>> обуславливает отличия в работе с объектами по сравнению с традиционными языками, использующими классы. Кроме того, JavaScript имеет ряд свойств, присущих функциональным языкам -- функции как значения, объекты как списки, карринг, анонимные функции, замыкания \cite{wiki:JS:ru}.

\section{Виртуальные машины JavaScript}

Осуществляя каких-либо оптимизации важно знать и учитывать особенности целевой платформы. Так сложилось, что для JavasScript не существует единой, общепризнанной виртуальной машины. Существует несколько виртуальных машин JavaScript получившие наибольшее распространение и покрывают порядка 90\% рынка \cite{StatCounter, w3schools}:
\begin{description}
\item[V8.] Виртуальная машина JavaScript с открытым исходным кодом разрабатывающаяся в компании Google с сентября 2008 года. Она используется в таких проектах как Chrome, Node.js, Android, и т. д. \cite{wiki:V8:en, wiki:V8:ru}.

\item[SpiderMonkey.] Первый в истории движок JavaScript, был написан Бренданом Айком во время его работы в Netscape Communications, а позднее сделан открытым \cite{wiki:SpiderMonkey:ru}. В настоящее время SpiderMonkey поддерживается Mozilla Foundation. 
Наиболее популярными программами которые используют данный <<движок>> являются Mozilla Firefox, Adobe Acrobat и Adobe Reader.

\item[Rhino.] Это еще одина виртуальная машина JavaScript от Mozilla Foundation. Проект является открытым и полностью написан на Java \cite{wiki:Rhino:ru}.

\item[Chakra.] Виртуальная машина JavaScript разрабатываемая в Microsoft и используется в Internet Explorer версии 9 и старше \cite{wiki:Chakra:en}.

\item[JavaScriptCore.] Встраиваемая виртуальная машина JavaScript с открытыми исходными кодами, является частью проекта WebKit. JavaScriptCore разрабатывается участниками проекта WebKit -- Apple, Google, BlackBerry, Adobe и другие. Используется в Safari и iOS \cite{JavaScriptCore}.

\end{description}


% \subsection{Выводы?}

% Выводы после изучения движков
% инициализировать все поля класса в конструкторе и в одном порядке
% нужно избегать мегаморфные вызовы
% нужно по возможности инициализировать массив сразу
% особенности оптимизации хранения чисел в v8
% избегать изменения типа переменной


\section{Язык программирования Kotlin}

Kotlin -- это статически типизированный объектно-ориентированный язык, разрабатываемый в компании JetBrains~\cite{KotlinOSP}.
Язык предназначен для промышленной разработки приложений. Компилируется в Java байт-код и JavaScript.
Создание такого языка -- это ответ на потребность в новом языке, которой был бы, с одной стороны, полностью совместим с Java, а с другой стороны, решал бы многочисленные проблемы, которые существуют в Java, но не могут быть исправлены по ряду технических причин.

При разработке, к новому языка предъявлялись следующие важные требования~\cite{KotlinOSP}:
\begin{itemize}
	\item Легкая совместимость с Java в обе стороны.
	\item Наличие статических гарантий корректности.
	\item Быстрая компиляция.
	\item Лаконичность и доступность для изучения.
\end{itemize}

% Kotlin имеет локоничный и выразительный синтаксис, в то же время остается простым для изучения. Как и любой стат


% Наряду со всем этим хочется чтобы компилятор оставался достаточно быстрым. Это одна из причин почему нам не подходят готовые решения вроде GWT, Doppio, и другие.



\section{Компиляция в JavaScript}
%todo нужно переименовать

Эксперты в один голос заявляют, что JavaScript сегодня стал ассемблером для веба \cite{JsIsAsm1, JsIsAsm2}.
Основной причиной таких высказываний является то, что за последние годы появилось много проектов целью которых является либо компиляция какого-либо зарекомендовавшего себя в индустрии языка в JavaScript, либо разработка нового языка который заменит JavaScript. Причем, наличие компилятора в JavaScript является обязательным атрибутом и проектов второй группы.
Все это еще раз доказывает актуальность данного направления разработок. И, очевидно, наличие высокого интереса приводит к повышению уровня конкуренции, что, в свою очередь, повышает требование к качеству продукта.

На сайте \url{altjs.org} представлен большой обзор проектов занимающихся компиляцией в JavaScript. Рассмотрим некотрые наиболее популярные сегодня проекты \cite{langpop}:
\begin{description}
\item[Компилятор GWT Java-to-JavaScript]
Это один из компонентов разрабатываемого в компании Google открытого Java-фреймворка Google Web Toolkit (GWT). С помощью данного компилятора выполняется компиляция Java кода в JavaScript.

\item[Язык программирования Dart]
Это объектно-ориентированный язык программирования разрабатываемая в компании Google, позиционирующийся как замена JavaScript в части разработки веб-приложений \cite{wiki:Dart:en}. % Разработчики из Google считают что имеющиеся в JavaScript «фундаментальных» изъяны невозможно исправить путём эволюционного развития. 
Код написанный на данном языке может быть запущен с использованием виртуальной машины DartVM. Или же, можно воспользоваться входящим в Dart SDK компилятором dart2js, который преобразует данный код в JavaScript и запустить в любой виртуальной машине JavaScript \cite{Dart}.

\item[Язык программирования CoffeeScript]
Это язык транслируемый в JavaScript. Разработчики языка постарались сохранить все сильные стороны языка JavaScript \cite{CoffeeScript}.

\item[Язык программирования TypeScript]
Статически типизированный язык программирования, разрабатываемый в Microsoft. Позиционируется как средство разработки веб-приложений, расширяющее возможности JavaScript. Объектная модель в языке реализована на основе классов \cite{wiki:TypeScript, TypeScript}.

\end{description}


\section{Выбор языков для сравнения}

Компиляция языка Kotlin в JavaScript с одной стороны, выглядит почти данью моде, а с другой -- находится в окружении заметного числа конкурентов. Поэтому хотелось бы не просто повысить качество компиляции, но и проследить за изменением положения компилятора Kotlin относительно конкурентов, как вначале, так в конце данного исследования. Для этого было решено отобрать несколько языков с которыми будет осуществляться постоянное сравнение. Одним из важных требований в данном выборе было наличие в открытом доступе тестов производительности. Это требование очень важно, т.к. оно избавляет от предвзятости при реализации данных тестов для конкурирующих языков. Так же, такое требование, скорее всего, позволит получить качественные тесты написанные экспертами в данном языке. 

Конечно же, в сравнении будет участвовать язык JavaScript. Для нашего компилятора это <<целевая платформа>> и было бы логичным сравнивать результат компиляции с <<вручную>> написанным кодом, тем более если этот код был написан экспертами. А для языка JavaScript существует большое количество тестов, написанных разработчиками виртуальных машин. В рамках данной работы было решено взять тесты производительности написанные разработчиками V8, так как эти тесты позиционируются как близкие к реальным приложениям \cite{V8:Benchmarks, Octane, Dromaeo}.

Так же, для сравнения был взят язык Dart. На момент начала исследования в открытом доступе было два теста производительности, написанных разработчиками языка. Предполагается сравнение результатов как тестов запущенных внутри DartVM, так и тестов, полученных компиляцией с помощью Dart2js и запущенных внутри виртуальной машины JavaScript.

GWT с компилятором Java-to-JavaScript не был взят для сравнения по двум причинам:
\begin{enumerate}
\item Данный компилятор оказался сильно связанным с остальными модулями библиотеки, что в свою очередь затруднет его использование отдельно от всех остальных модулей. Это, в свою очередь, усложняет проведение точных тестов и проведение тестов вне браузерного окружения.
\item Специально написанных для GWT тестов обнаружено не было. И как показала практика, не любой тест, написанный для Java и работающий хорошо в JVM-окружении, будет хорошо работать после компиляции с помощью данного компилятора.
\end{enumerate}

Для CoffeeScript и TypeScript таких тестов тоже не оказалось. Это во многом объясняется тем, что данные языки не очень далеко <<ушли>> от JavaScript, и тем как данные языки себя позиционируют. Конструкции обоих языков легко проецируются на конструкции JavsScript. А TypeScript вообще является надмножеством языка JavaScript и обратно совместим с ним. Все это позволяет разработчикам данных языков забыть о тестах производительности \cite{CoffeeScript}.

Несмотря на то, что некоторые проекты, по тем или иным причинам, для сравнения нам не подошли, они по прежнему остаются для данной работы источниками ценного опыта в данной области.