\chapter{Обзор}

\section{Язык программирования JavaScript}

JavaScript -- это язык программирования для Веб. Подавляющее большинство веб-сайтов используют JavaScript, и все современные веб-браузеры -- для настольных компьютеров, игровые приставок, электронных планшетов и смартфонов включают интерпретатор JavaScript, что делает JavaScript самым широкоприменимым языком программирования из когда либо существовавших в истории. JavaScript входит в тройку технологий, которые должны знать  любой веб-разработчик: язык разметки HTML, позволяющий определить содержимое веб-страниц, язык стилей CSS, позволяющий определить внешний вид веб-страниц, и язык программирования JavaScript, позволяющий определять поведение веб-страниц.

JavaScript является высокоуровневым, динамическим, нетипизрованым и интрепретируемым языком программирования, который хорошо подходит для программирования в объектно-ориентированном и функциональном стилях. Свой синтаксис JavaScript унаследовал из языка Java, свои первоклассные функции -- из языка Scheme, а механизм наследования на основе прототипов -- из языка Self. 

Название языка «JavaScript» может вводить в заблуждение. За исключением поверхностной синтаксической схожести, JavaScript полностью отличается от языка программирования Java. JavaScript давно перерос рамки языка сценариев, превратившись в надежный и эффективный универсальный язык программирования. Последняя версия языка (смотрите врезку) определяет множество новых особенностей, позволяющих использовать его для разработки крупномасштабного программного обеспечения.
% Чтобы представлять хоть какой-то интерес, каждый язык программирования должен иметь свою платформу, или стандартную библиотеку или API функций для выполнения таких базовых операций, как ввод и  вывод. Ядро языка JavaScript определяет минимальный прикладной интерфейс для работы с текстом, массивами, датами и регулярными выражениями, но в нем отсутствуют операции ввода-вывода. Ввод и вывод (а также более сложные возможности, такие как сетевые взаимодействия, сохранение данных и работа с графикой) перекладываются на «окружающую среду», куда встраивается JavaScript. Обычно роль окружающей среды играет веб-броузер.
\cite{JsDef6}

JavaScript является объектно-ориентированным языком, но используемое в языке прототипирование обуславливает отличия в работе с объектами по сравнению с традиционными класс-ориентированными языками. Кроме того, JavaScript имеет ряд свойств, присущих функциональным языкам — функции как объекты первого класса, объекты как списки, карринг, анонимные функции, замыкания.
\cite{wiki:JS:ru}

\section{JavaScript виртуальные машины}
 
\subsection{SpiderMonkey}

SpiderMonkey -- первый в истории движок JavaScript, был написан Бренданом Айком во время его работы в Netscape Communications, а позднее сделан открытым. В настоящее время SpiderMonkey поддерживается Mozilla Foundation. Он написан на языке Си и включает в себя компилятор, интерпретатор, декомпилятор, сборщик мусора и стандартные классы. SpiderMonkey встраивается в другие приложения, которые предоставляют рабочее окружение для JavaScript.

Наиболее популярными программами которые используют данный <<движок>> являются Mozilla Firefox и Mozilla Application Suite/SeaMonkey, а также Adobe Acrobat и Adobe Reader.
\cite{wiki:SpiderMonkey:ru}

\subsection{Rhino}

Еще одина JavaScript виртуальная машина от Mozilla Foundation. Проект является открытым и полностью написан на Java. Rhino преобразует JavaScript код в Java классы. Движок работает и в компилируемом и интерпретируемом режимах. Он предназначен для использования в server-side приложениях, поэтому в нём нет встроенной поддержки для объектов браузера, которые обычно ассоциируются с JavaScript.
\cite{wiki:Rhino:ru}

\subsection{V8}
JavaScript виртуальная машина с открытым исходным кодом разрабатывающаяся в компании Google с сентября 2008 года. Отличительной особенностью данного проекта является то что код компилируется непосредственно в машинный код без использования промежуточного представления ввиде байт-кода. Виртуальная машина поддерживает архитектуры IA-32, x86-64, ARM, и MIPS. Обладает эффективной системой управления памятью, приводящая к быстрому объектному выделению и маленьким паузам сборки <<мусора>>. В данном <<движке>> активно используется так называемые <<скрытые классы>> и встроенные кэши, ускоряющие доступ к свойствам и вызовы функций.

V8 используется в таких проектах как Chrome, Node.js, Android, и т. д.
\cite{wiki:V8:en, wiki:V8:ru}

\subsection{Chakra}
JavaScript виртуальная машина разрабатываемая в Microsoft. Отличительной особенностью данной виртуальной машины от большинства является наличие нескольких независимых модулей, которые могут работать параллельно друг другу и параллельно модулю отрисовки браузера. Это такие модули как: компилятор JavaScript кода в байт-код, JIT-компилятор, сборщик мусора. Так же движок при необходимости использует возможности графической платы.
Chakra используется в Internet Explorer версии 9 и старше.
\cite{wiki:Chakra:en}

\subsection{JavaScriptCore}

Встраиваемая JavaScript виртуальная машина с открытыми исходными кодами, является частью проекта WebKit. JavaScriptCore разрабатывается участниками проекта WebKit -- Apple, Google, BlackBerry, Adobe и другие. Кроме стандартного набора интерпретатор, сборщик мусора он так же включает в себя два JIT компилятора -- базовый(быстрый) и оптимизирующий.
Используется в Safari и iOS.
\cite{JavaScriptCore}


\section{Язык программирования Kotlin}

Kotlin — статически типизированный объектно-ориентированный язык, разрабатываемый в компании JetBrains.
Есть компиляторы в Java байт-код и в JavaScript.
Kotlin имеет локоничный и выразительный синтаксис, в то же время остается простым для изучения. Как и любой стат

Наряду со всем этим хочется чтобы компилятор оставался достаточно быстрым. Это одна из причин почему нам не подходят готовые решения вроде GWT, Doppio, и другие.

\cite{KotlinOSP}

\section{Компиляторы в JavaScript}
%todo нужно переименовать

Эксперты в один голос заявляют, что JavaScript сегодня стал ассемблером для веба.\cite{JsIsAsm1, JsIsAsm2}
Основной причиной таких высказываний является то, что за последние годы появилось много проектов целью которых является либо компиляция какого-либо зарекомендовавшего себя в индустрии языка в JavaScript, либо разработка нового языка который заменит JavaScript. Причем, наличие компилятора в JavaScript является обязательным атрибутом и проектов второй группы.
Все это еще раз доказывает актуальность данного направления разработок. И, очевидно, наличие высокого интереса приводит к повышению уровня конкуренции, что, в свою очередь, повышает требование к качеству продукта.

На сайте \url{altjs.org} представлен большой обзор проектов занимающихся компиляцией в JavaScript. Рассмотрим некотрые наиболее популярные сегодня проекты.
\cite{langpop}

\subsection{Компилятор GWT Java-to-JavaScript}

Google Web Toolkit (GWT) -- это  разработанный в компании Google открытый Java-фреймворк, который позволяет разработчикам создавать веб-приложения используя язык программирования Java. GWT делает акцент на повторное использование и кросс-браузерную совместимость.
Используя GWT, разработчики могут быстро писать и отлаживать веб-приложения на языке Java, используя инструментарий отладки Java. Компилятор GWT переведёт код Java приложения в соответствующий браузеру JavaScript и HTML.
\cite{wiki:GWT:ru}
GWT используется во многих продуктах компании Google, например, в  Google Wave и AdWords.
\cite{GWT:overview}

Компилятор GWT Java-to-JavaScript это один из компонентов GWT, который выполняет компиляцию Java кода в JavaScript.

% К преимуществам разработки на Google Web Toolkit по сравнению с
% JavaScript можно отнести:
% - Простоту в освоении для Java – программистов.
% - Отличную поддержку в современных IDE языка Java и каркаса GWT.
% - Возможность использовать значительную часть стандартных библиотек Java.
% - Богатую библиотеку элементов интерфейса.
% - Оптимизирующий генератор кода.
% - Возможность отладки прямо в браузере.
% К недостаткам можно отнести:
% - Невысокую скорость генерации JavaScript.
% - Сложный механизм взаимодействия с библиотеками, написанными на
% JavaScript.
% - Невозможность безболезненно отказаться от использования GWT (перейти к
% использованию другой технологии без переписывания значительной части
% кода).

\subsection{Язык программирования Dart}

Dart -- объектно-ориентированный язык программирования разрабатываемая в компании Google. Исходные коды данного проекта находятся в открытом доступе. Dart позиционируется в качестве замены JavaScript в части разработки веб-приложений. Разработчики из Google считают что имеющиеся в JavaScript «фундаментальных» изъяны невозможно исправить путём эволюционного развития. В отличие от языка JavaScript реализация объектно-ориентированной парадигмы в языке Dart базируется на классах. Так же, в коде, написанном на данном языке, может присутствовать опциональная типизация.
\cite{wiki:Dart:en}
Задачи, поставленные перед разработчиками языка:\cite{Dart}
\begin{itemize}
\item Создать структурированный и в то же время гибкий язык для веб-программирования.
\item Сделать язык похожим на существующие для упрощения обучения.
\item Высокая производительность получаемых программ как в браузерах, так и в иных окружениях, начиная от смартфонов и заканчивая серверами.
\end{itemize}

Код написанный на данном языке может быть запущен с использованием виртуальной машины DartVM или же, можно воспользоваться входящим в Dart SDK компилятором dart2js, который преобразует данный код в JavaScript.\cite{Dart}


\subsection{Язык программирования CoffeeScript}
%todo отредактировать

CoffeeScript -- язык программирования, транслируемый в JavaScript. Разработчики языка постарались сохранить все сильные стороны языка JavaScript. Так же в язык было добавлено множество синтаксического сахара в духе Ruby, Python, Haskell и Erlang для того, чтобы улучшить читаемость кода и уменьшить его размер. В среднем для выполнения одинаковых действий на CoffeeScript требуется в 2 раза меньше строк, чем JavaScript.

Компилятор CoffeeScript чистый и лаконичный JavaScript код, который полностью проходит проверку JavaScript Lint.
\cite{CoffeeScript}
% За небольшое время своего существования CoffeeScript (версия 0.1 выпущена в
% конце 2009 года, в мае 2012 года выпущена версия 1.3) приобрел популярность, благодаря компактному и выразительному синтаксису, а также легкости в изучении для программистов, знакомых с JavaScript.
% К преимуществам разработки на языке CoffeeScript по сравнению с JavaScript можно
% отнести:
% - Выразительный синтаксис.
% - Поддержку классов.

\subsection{Язык программирования TypeScript}
%todo отредактировать

TypeScript -- статически типизированный язык программирования разрабатываемый в Microsoft. Позиционируется как средство разработки веб-приложений, расширяющее возможности JavaScript. Объектная модель в язык реализована на основе классов.

Разработчиком языка TypeScript является Андерс Хейлсберг создавший ранее Turbo Pascal, Delphi и C\#.
TypeScript является обратно совместимым с JavaScript и компилируется в последний. Фактически, после компиляции программу на TypeScript можно выполнять в любом современном браузере или использовать совместно с серверной платформой Node.js.

TypeScript отличается от JavaScript возможностью явного определения типов (статическая типизация), поддержкой использования полноценных классов (как в традиционных объектно-ориентированных языках), а также поддержкой подключения модулей. По идее, подобные нововведения должны повысить скорость разработки, читабельность, рефакторинг и повторное использования кода, осуществлять поиск ошибок на этапе разработки и компиляции, а также скорость выполнения программ.
Предполагается, что в силу полной обратной совместимости адаптация существующих приложений на новый язык программирования может происходить поэтапно, путём постепенного определения типов. \cite{wiki:TypeScript, TypeScript}

\section{Выбор языков для сравнения}

хочется не только понимать что мы улучшились хочется и посмотреть как мы выглядим на фоне остальных

С кем сравниваемся?

JavaScript (бенчмарки от разработчиков v8)
Dart (бенчмарки от разработчиков)
Dart2js

Google Web Toolkit
в паблике тестов производительности нет

Dart -- есть

CoffeeScript и TypeScript нет, во многом это объясняется тем, что эти языки не сильно отличаются от JS ??? у них нету кросс библиотек которые нужно поддерживать на разных платформах. Система типов либо вообще не отличается либо незначительно. Более просетое наследование.

Сравниваться со всем кончено получилось бы и пришлось выбрать нескольких.
Почему именно эти аналоги?
Ну во первых это реализация на чистом JS, были взяты тесты от разработчиков v8. Я считаю это обязательным пунктом, чтобы показать, понять на сколько мы хуже чем код написанный вручную на целевом языке.(тем более экспертами) И сделать выводы в чем мы хуже и почему, что можно улучшить. А в итого показать на сколько мы хуже или лучше чем такой код.

Казалось бы что может быть лучше вручную написанного JS кода? Я решил немного повысить планку и взять Dart, особенность в том, что дарт спроектирован с учетом некоторых недостатков JS, язык разрабатывается практически той же коммандой которая разрабатывает v8, нативная реализация дарта выполняется в своей виртуальной машине которая знает о коде чуть больше и очень активно пользуется этим, за счет чего на многих тестах уже сейчас показывает лучшие результаты чем вручную написанный JS

Ну и планачка чуть по ниже это js полученный путем трансляции дарт в js

ТО план минимум работать так же хорошо как dart2js, план максимум работать так же хорошо как dart(видимо не всегда достижимый)


НУЖНЫ ВЫВОДЫ!!!
