\chapter{Обзор}

\section{Язык программирования JavaScript}

JavaScript -- это язык программирования для веб. Подавляющее большинство веб-сайтов используют JavaScript, и все современные веб-браузеры -- для настольных компьютеров, игровые приставок, электронных планшетов и смартфонов включают интерпретатор JavaScript, что делает JavaScript самым широкоприменимым языком программирования из когда либо существовавших в истории. JavaScript входит в тройку технологий, которые должны знать  любой веб-разработчик: язык разметки HTML, позволяющий определить содержимое веб-страниц, язык стилей CSS, позволяющий определить внешний вид веб-страниц, и язык программирования JavaScript, позволяющий определять поведение веб-страниц.

JavaScript является высокоуровневым, динамическим, нетипизированным и интерпретируемым языком программирования, который хорошо подходит для программирования в объектно-ориентированном и функциональном стилях. Свой синтаксис JavaScript унаследовал из языка Java, свои первоклассные функции -- из языка Scheme, а механизм наследования на основе прототипов -- из языка Self. 

Название языка «JavaScript» может вводить в заблуждение. За исключением поверхностной синтаксической схожести, JavaScript полностью отличается от языка программирования Java. JavaScript давно перерос рамки языка сценариев, превратившись в надежный и эффективный универсальный язык программирования. Последняя версия языка (смотрите врезку) определяет множество новых особенностей, позволяющих использовать его для разработки крупномасштабного программного обеспечения.
% Чтобы представлять хоть какой-то интерес, каждый язык программирования должен иметь свою платформу, или стандартную библиотеку или API функций для выполнения таких базовых операций, как ввод и  вывод. Ядро языка JavaScript определяет минимальный прикладной интерфейс для работы с текстом, массивами, датами и регулярными выражениями, но в нем отсутствуют операции ввода-вывода. Ввод и вывод (а также более сложные возможности, такие как сетевые взаимодействия, сохранение данных и работа с графикой) перекладываются на «окружающую среду», куда встраивается JavaScript. Обычно роль окружающей среды играет веб-броузер.
\cite{JsDef6}

JavaScript является объектно-ориентированным языком, но используемое в языке прототипирование обуславливает отличия в работе с объектами по сравнению с традиционными класс-ориентированными языками. Кроме того, JavaScript имеет ряд свойств, присущих функциональным языкам -- функции как объекты первого класса, объекты как списки, карринг, анонимные функции, замыкания.
\cite{wiki:JS:ru}

\section{JavaScript виртуальные машины}
 
\subsection{SpiderMonkey}

SpiderMonkey -- первый в истории движок JavaScript, был написан Бренданом Айком во время его работы в Netscape Communications, а позднее сделан открытым. В настоящее время SpiderMonkey поддерживается Mozilla Foundation. Он написан на языке Си и включает в себя компилятор, интерпретатор, декомпилятор, сборщик мусора и стандартные классы. SpiderMonkey встраивается в другие приложения, которые предоставляют рабочее окружение для JavaScript.

Наиболее популярными программами которые используют данный <<движок>> являются Mozilla Firefox и Mozilla Application Suite/SeaMonkey, а также Adobe Acrobat и Adobe Reader.
\cite{wiki:SpiderMonkey:ru}

\subsection{Rhino}

Еще одина JavaScript виртуальная машина от Mozilla Foundation. Проект является открытым и полностью написан на Java. Rhino преобразует JavaScript код в Java классы. Движок работает и в компилируемом и интерпретируемом режимах. Он предназначен для использования в server-side приложениях, поэтому в нём нет встроенной поддержки для объектов браузера, которые обычно ассоциируются с JavaScript.
\cite{wiki:Rhino:ru}

\subsection{V8}
JavaScript виртуальная машина с открытым исходным кодом разрабатывающаяся в компании Google с сентября 2008 года. Отличительной особенностью данного проекта является то что код компилируется непосредственно в машинный код без использования промежуточного представления ввиде байт-кода. Виртуальная машина поддерживает архитектуры IA-32, x86-64, ARM, и MIPS. Обладает эффективной системой управления памятью, приводящая к быстрому объектному выделению и маленьким паузам сборки <<мусора>>. В данном <<движке>> активно используется так называемые <<скрытые классы>> и встроенные кэши, ускоряющие доступ к свойствам и вызовы функций.

V8 используется в таких проектах как Chrome, Node.js, Android, и т. д.
\cite{wiki:V8:en, wiki:V8:ru}

\subsection{Chakra}
JavaScript виртуальная машина разрабатываемая в Microsoft. Отличительной особенностью данной виртуальной машины от большинства является наличие нескольких независимых модулей, которые могут работать параллельно друг другу и параллельно модулю отрисовки браузера. Это такие модули как: компилятор JavaScript кода в байт-код, JIT-компилятор, сборщик мусора. Так же движок при необходимости использует возможности графической платы.
Chakra используется в Internet Explorer версии 9 и старше.
\cite{wiki:Chakra:en}

\subsection{JavaScriptCore}

Встраиваемая JavaScript виртуальная машина с открытыми исходными кодами, является частью проекта WebKit. JavaScriptCore разрабатывается участниками проекта WebKit -- Apple, Google, BlackBerry, Adobe и другие. Кроме стандартного набора интерпретатор, сборщик мусора он так же включает в себя два JIT компилятора -- базовый(быстрый) и оптимизирующий.
Используется в Safari и iOS.
\cite{JavaScriptCore}

% \subsection{Выводы?}

% Выводы после изучения движков
% инициализировать все поля класса в конструкторе и в одном порядке
% нужно избегать мегаморфные вызовы
% нужно по возможности инициализировать массив сразу
% особенности оптимизации хранения чисел в v8
% избегать изменения типа переменной


\section{Язык программирования Kotlin}

Kotlin -- это статически типизированный объектно-ориентированный язык, разрабатываемый в компании JetBrains.
Язык предназначен для промышленной разработки приложений. Компилируется в Java байт-код и JavaScript.
Создание такого языка -- это ответ на потребность в новом языке, которой был бы, с одной стороны, полностью совместим с языком Java, а, с другой стороны, решал бы многочисленные проблемы, которые существуют в языке Java, но не могут быть исправлены по ряду технических причин.

При создании языка учитывались некоторые важные требования~\cite{KotlinOSP}:
\begin{description}
	\item[Совместимость с Java.] Платформа Java~--- это прежде всего экосистема: существует множество продуктов и библиотек, на базе которых строится огромное количество приложений.
	Поэтому для нового языка очень важна совместимость с уже существующим кодом. Важным моментом является тот факт, что миграция на новый язык может происходить постепенно, таким образом, не только код на Kotlin должен легко вызывать код на Java, но и наоборот.

	\item[Статические гарантии корректности.] Во время компиляции кода на статически типизированном языке происходит множество проверок, призванных гарантировать, что те или иные ошибки не произойдут во время выполнения.
	 Ярким примером такой исключительной ситуации является разыменование нулевой ссылки. Важным требованием к новому языку является усиление статических гарантий. Это позволит обнаруживать больше ошибок на этапе компиляции и, таким образом, сокращать затраты на тестирование.

	\item[Скорость компиляции.] Статические проверки упрощают программирование, но замедляют компиляцию, и здесь необходимо добиться определенного баланса. Опыт создания языков с мощной системой типов (яркими примерами таких языков являются Scala~\cite{scala-spec}, Haskell~\cite{haskell98}) показывает, что такой баланс найти непросто: компиляция зачастую становится неприемлемо долгой.
	Вообще, такая характеристика языка, как время компиляции проекта, может показаться второстепенной, однако в условиях когда объемы компилируемого кода очень велики, оказывается, что этот фактор весьма важен~--- ведь пока код компилируется, программист зачастую не может продолжать работу. Известным примером медленной компиляции является язык C++.

	\item[Лаконичность.] Известно~\cite{codecomplete}, что программисты зачастую тратят больше времени на чтение кода, чем на его написание, поэтому важно, чтобы конструкции, доступные в языке программирования, позволяли писать программы кратко и понятно.
	Java считается многословным языком (ceremony language~--- <<церемонный язык>>), и задача Kotlin~--- улучшить ситуацию в этом смысле.

	\item[Доступность для изучения.] Сложные статические проверки, гибкий синтаксис и конструкции высших порядков усложняют язык и затрудняют его изучение, поэтому необходимо в известной степени ограничивать набор поддерживаемых возможностей, чтобы язык был доступен для изучения. При разработке Kotlin учитывался опыт создания других современных языков, и слишком сложные концепции в язык не включались.

	\item[Инструментальная поддержка.] Важным условием успешности языка программирования является наличие хорошей инструментальной поддержки. Центральное место среди таких инструментов занимают интегрированные среды разработки (Integrated Development Environment, IDE).
	При разработке Kotlin, IDE создается одновременно с компилятором. Такой подход позволяет одновременно решить сразу несколько проблем: модифицировать или удалить из языка концепции, которые сложны в поддержке со стороны среды разработки, а также предоставить хорошую инфраструктуру для прикладных программистов с первых дней выпуска продукта.
\end{description}

% Kotlin имеет локоничный и выразительный синтаксис, в то же время остается простым для изучения. Как и любой стат


% Наряду со всем этим хочется чтобы компилятор оставался достаточно быстрым. Это одна из причин почему нам не подходят готовые решения вроде GWT, Doppio, и другие.



\section{Компиляция в JavaScript}
%todo нужно переименовать

Эксперты в один голос заявляют, что JavaScript сегодня стал ассемблером для веба.\cite{JsIsAsm1, JsIsAsm2}
Основной причиной таких высказываний является то, что за последние годы появилось много проектов целью которых является либо компиляция какого-либо зарекомендовавшего себя в индустрии языка в JavaScript, либо разработка нового языка который заменит JavaScript. Причем, наличие компилятора в JavaScript является обязательным атрибутом и проектов второй группы.
Все это еще раз доказывает актуальность данного направления разработок. И, очевидно, наличие высокого интереса приводит к повышению уровня конкуренции, что, в свою очередь, повышает требование к качеству продукта.

На сайте \url{altjs.org} представлен большой обзор проектов занимающихся компиляцией в JavaScript. Рассмотрим некотрые наиболее популярные сегодня проекты.
\cite{langpop}

\subsection{Компилятор GWT Java-to-JavaScript}

Google Web Toolkit (GWT) -- это  разработанный в компании Google открытый Java-фреймворк, который позволяет разработчикам создавать веб-приложения используя язык программирования Java. GWT делает акцент на повторное использование и кросс-браузерную совместимость.
Используя GWT, разработчики могут быстро писать и отлаживать веб-приложения на языке Java, используя инструментарий отладки Java. Компилятор GWT переведёт код Java приложения в соответствующий браузеру JavaScript и HTML.
\cite{wiki:GWT:ru}
GWT используется во многих продуктах компании Google, например, в  Google Wave и AdWords.
\cite{GWT:overview}

Компилятор GWT Java-to-JavaScript это один из компонентов GWT, который выполняет компиляцию Java кода в JavaScript.

% К преимуществам разработки на Google Web Toolkit по сравнению с
% JavaScript можно отнести:
% - Простоту в освоении для Java – программистов.
% - Отличную поддержку в современных IDE языка Java и каркаса GWT.
% - Возможность использовать значительную часть стандартных библиотек Java.
% - Богатую библиотеку элементов интерфейса.
% - Оптимизирующий генератор кода.
% - Возможность отладки прямо в браузере.
% К недостаткам можно отнести:
% - Невысокую скорость генерации JavaScript.
% - Сложный механизм взаимодействия с библиотеками, написанными на
% JavaScript.
% - Невозможность безболезненно отказаться от использования GWT (перейти к
% использованию другой технологии без переписывания значительной части
% кода).

\subsection{Язык программирования Dart}

Dart -- объектно-ориентированный язык программирования разрабатываемая в компании Google. Исходные коды данного проекта находятся в открытом доступе. Dart позиционируется в качестве замены JavaScript в части разработки веб-приложений. Разработчики из Google считают что имеющиеся в JavaScript «фундаментальных» изъяны невозможно исправить путём эволюционного развития. В отличие от языка JavaScript реализация объектно-ориентированной парадигмы в языке Dart базируется на классах. Так же, в коде, написанном на данном языке, может присутствовать опциональная типизация.
\cite{wiki:Dart:en}
Задачи, поставленные перед разработчиками языка:\cite{Dart}
\begin{itemize}
\item Создать структурированный и в то же время гибкий язык для веб-программирования.
\item Сделать язык похожим на существующие для упрощения обучения.
\item Высокая производительность получаемых программ как в браузерах, так и в иных окружениях, начиная от смартфонов и заканчивая серверами.
\end{itemize}

Код написанный на данном языке может быть запущен с использованием виртуальной машины DartVM или же, можно воспользоваться входящим в Dart SDK компилятором dart2js, который преобразует данный код в JavaScript.\cite{Dart}


\subsection{Язык программирования CoffeeScript}
%todo отредактировать

CoffeeScript -- язык программирования, транслируемый в JavaScript. Разработчики языка постарались сохранить все сильные стороны языка JavaScript. Так же в язык было добавлено множество синтаксического сахара в духе Ruby, Python, Haskell и Erlang для того, чтобы улучшить читаемость кода и уменьшить его размер. В среднем для выполнения одинаковых действий на CoffeeScript требуется в 2 раза меньше строк, чем JavaScript.

Компилятор CoffeeScript генерирует чистый и лаконичный JavaScript код, который полностью проходит проверку JavaScript Lint.
\cite{CoffeeScript}
% За небольшое время своего существования CoffeeScript (версия 0.1 выпущена в
% конце 2009 года, в мае 2012 года выпущена версия 1.3) приобрел популярность, благодаря компактному и выразительному синтаксису, а также легкости в изучении для программистов, знакомых с JavaScript.
% К преимуществам разработки на языке CoffeeScript по сравнению с JavaScript можно
% отнести:
% - Выразительный синтаксис.
% - Поддержку классов.

\subsection{Язык программирования TypeScript}
%todo отредактировать

TypeScript -- статически типизированный язык программирования разрабатываемый в Microsoft. Позиционируется как средство разработки веб-приложений, расширяющее возможности JavaScript. Объектная модель в язык реализована на основе классов.

Разработчиком языка TypeScript является Андерс Хейлсберг создавший ранее Turbo Pascal, Delphi и C\#.
TypeScript является обратно совместимым с JavaScript и компилируется в последний. Фактически, после компиляции программу на TypeScript можно выполнять в любом современном браузере или использовать совместно с серверной платформой Node.js.

TypeScript отличается от JavaScript возможностью явного определения типов (статическая типизация), поддержкой использования полноценных классов (как в традиционных объектно-ориентированных языках), а также поддержкой подключения модулей. По идее, подобные нововведения должны повысить скорость разработки, читабельность, рефакторинг и повторное использования кода, осуществлять поиск ошибок на этапе разработки и компиляции, а также скорость выполнения программ.
Предполагается, что в силу полной обратной совместимости адаптация существующих приложений на новый язык программирования может происходить поэтапно, путём постепенного определения типов. \cite{wiki:TypeScript, TypeScript}

\section{Выбор языков для сравнения}

Компиляция языка Kotlin в JavaScript с одной стороны, выглядит почти данью моде, а с другой -- находится в окружении заметного числа конкурентов. Поэтому хотелось бы не просто повысить качестве компиляции, но и так же было бы не плохо понимать каково наше положение относительно конкурентов как в начале, так и в конце данного исследования. Для этого было решено отобрать несколько языков с которыми будет осуществляться постоянное сравнивание. Одним из важных требований в данном выборе было наличие в открытом доступе тестов производительности. Это требование очень важно, т.к. оно избавляет предвзятости при реализации данных тестов для конкурирующих языков. Так же, такое требование, скорее всего, позволит получить качественные тесты написанные экспертами в данном языке. 

Конечно же в сравнении будет участвовать язык JavaScript. Во-первых это целевая платформа и было бы логичным сравнивать результат компиляции с <<вручную>> написанным кодом, тем более если этот код был написан экспертами. А для языка JavaScript существует большое количество тестов написанные разработчиками виртуальны машин. В рамках данной работы было решено взять тесты производительности написанные разработчиками V8, так как эти тесты позиционируются как близкие к реальным приложениям.

Так же, для сравнения был взят язык Dart. На момент начало исследования в открытом доступе было 2 теста производительности написанные разработчиками языка. Предполагается сравнивание результатов как тестов запущенных внутри DartVM, так и тестов полученных компиляцией с помощью Dart2js и запущенных внутри виртуальной машины JavaScript.

Остальные проекты по тем или иным причинам для сравнения нам не подошли, но они по прежнему остаются для данной работы источниками ценного опыта в данной области.

GWT с компилятором Java-to-JavaScript не был взят для сравнения по двум причинам:
\begin{enumerate}
\item Данный компилятор оказался сильно связанным с остальными модулями библиотеки, что в свою очередь затруднет его использование отдельно от всех остальных модулей. Это, в свою очередь, усложняет проведение точных тестов и проведение тестов вне браузерного окружения.
\item Специально написанных для GWT обнаружено не было. И как показала практика не любой тест написанный для Java и работающий хорошо в JVM окружении будет хорошо работать после компиляции с помощью данного компилятора.
\end{enumerate}

Для CoffeeScript и TypeScript таких тестов тоже не оказалось. Это во много объясняется тем, что данные языки не очень далеко <<ушли>> от JavaScript и тем как данные языки себя позиционируют. Конструкции обоих языков легко проецируются на конструкции языка JavsScript. А TypeScript вообще является надмножеством языка JavaScript и обратно совместим с ним. Все это позволяет разработчикам данных языков забыть о тестах производительности.