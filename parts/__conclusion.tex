\startconclusionpage
%\section { Заключаение}

% Квалификационной работа завершается заключительной частью, которая так и называется «заключение». В заключении должно быть дано последовательное, логически стройное изложение полученных результатов и указано их соотношение с общей целью и конкретными задачами, сформулированными во введении. Именно здесь содержится, так называемое, «выводное» знание, которое является новым по отношению к исходному знанию и выносится на обсуждение и оценку в процессе защиты работы.
% Заключительная часть предполагает также наличие обобщенной итоговой оценки проделанной работы с точки зрения ее новизны, теоретической и практической значимости. При этом важно указать, в чем заключается главный смысл работы, какие важные побочные результаты получены, какие встают новые задачи в связи с проведением исследования. 
% Заключительная часть, составленная по такому плану, дополняет характеристику уровня профессиональной зрелости и квалификации автора работы.

В рамках данной работы был проведен анализ качества кода, генерируемого компилятором Kotlin в JavaScript, с целью улучшения его производительности. Были выявлены узкие <<места>> генерируемого кода и предложены способы их устранения. Кроме того были реализованы некоторые из предложенных оптимизаций.


Автор планирует и дальше работать над этой задачей. В планах на будущее:
\begin{itemize}
\item Продолжить оптимизацию генерируемого кода.
\item Написать новые бенчмарки. 
\item Сделать чтобы генеририруемый код был более идеомотичным для JavaScript.
\item Добавить поддержку технологии source-map для удобной отладки сгенерированного кода.
\item Реализовать отсутствующие JavaScript компиляторе языковые фичи.
\item Оптимизации на основе Continuation-passing style и 
Static single assignment
\end{itemize}