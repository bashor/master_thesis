\startconclusionpage

В рамках данной работы был проведен анализ качества кода, генерируемого компилятором Kotlin в JavaScript, с целью улучшения его производительности. Были выявлены узкие <<места>> генерируемого кода и предложены способы их устранения. Кроме того, были реализованы некоторые из предложенных оптимизаций, благодаря которым код скомпилированный с помощью компилятора Kotlin стал выдавать результаты сравнимые с результатами Dart2js. Реализованные оптимизации:
\begin{itemize}
\item Оптимизирован доступ к свойствам.
%, для свойств с акссесорами используются свойства в стиле ES5
\item Улучшена реализация наследования.
%При конструировании больше не используются промежуточные функции.
\item Реализован <<прямой>> вызов супер конструктора.
%(похожие фиксы делал и для ES5)
\item Для сравнения примитивных типов используется оператор \path{===} вместо функции \path{Kotlin.equals}.
\item Оптимизирована функция \path{Kotlin.equals}, используемая для сравнения.
 %Еще есть кое какие идеи, которые нужно додумать, и реализовать.(но не сейчас)
\item Было оптимизировано вычисление хеш-кода (функция \path{hashObject}).
\item В HashMap и PrimitiveHashMap добавлена возможность использования своей функции хеширования.
\item Для примитивных типов используется PrimitiveHashMap.
\end{itemize}

\null
%todo
Результаты данной работы используются в официальной реализации компилятора Kotlin. \cite{KotlinORG}.

\null
Оптимизации, которые пока не удалось реализовать:

\begin{itemize}
\item Не удалось реализовать аналог импорта или предоставить оптимальный доступ хотя бы внутри пакета, поскольку не удалось найти достаточно оптимального решения для текущей структуры генерируемого кода. %Нужно  либо делать костыли, либо существенно менять структуру генерируемого кода.

\item Для случая когда в качестве ключа используется закрытый класс, в котором не определена своя реализация функции \path{hashCode}, не удалось реализовать использование \path{PrimitiveHashMap}.
\end{itemize}

\null
\begin{LARGE}
\begin{center}
\textbf{Направления дальнейших работ}
\end{center}
\end{LARGE}

Работа над компилятором Kotlin в JavaScript еще далека от завершающего этапа и в будущем, для его улучшения, планируется выполнение следующих работ:
\begin{itemize}
\item Пересмотреть реализации контейнеров в стандартной библиотеке.
\item Продолжить оптимизацию генерируемого кода.
\item Написать новые бенчмарки. 
\item Сделать чтобы генерируемый код был более идеомотичным для JavaScript (улучшить структуру генерируемого кода).
\item Добавить поддержку технологии source-map для удобной отладки сгенерированного кода.
\item Реализовать конструкции языка Kotlin отсутствующие в JavaScript компиляторе.
\item Реализовать оптимизации на основе Continuation-passing style и 
Static single assignment.
\end{itemize}