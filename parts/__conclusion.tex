\subsection{Выводы}

В рамках данной работы был проведен анализ качества кода, генерируемого компилятором Kotlin в JavaScript, с целью улучшения его производительности. Были выявлены узкие <<места>> генерируемого кода и предложены способы их устранения. Кроме того, были реализованы некоторые из предложенных оптимизаций, благодаря которым код скомпилированный с помощью компилятора Kotlin стал выдавать результаты сравнимые с результатами Dart2js. Реализованные оптимизации:
\begin{itemize}
\item Оптимизирован доступ к свойствам.
%, для свойств с акссесорами используются свойства в стиле ES5
\item Улучшена реализация наследования.
%При конструировании больше не используются промежуточные функции.
\item Реализован <<прямой>> вызов супер конструктора.
%(похожие фиксы делал и для ES5)
\item Для сравнения примитивных типов используется оператор \path{===} вместо функции \path{Kotlin.equals}.
\item Оптимизирована функция \path{Kotlin.equals}, использующая для сравнения.
 %Еще есть кое какие идеи, которые нужно додумать, и реализовать.(но не сейчас)
\item Было оптимизировано вычисление хеш-кода (функция \path{hashObject}).
\item В HashMap и PrimitiveHashMap добавлена возможность использования своей функции хешировани.
\item Для примитивных типов используется PrimitiveHashMap.
\end{itemize}

Оптимизации которые пока не получилось реализовать:
\begin{itemize}
\item Не получилось реализовать аналог импорта или предоставить оптимальный доступ хотя бы внутри пакета, так как достаточно хорошего решения для текущей структуры генерируемого код придумать не удалось. %Нужно  либо делать костыли, либо существенно менять структуру генерируемого кода.

\item Не получилось реализовать использование \path{PrimitiveHashMap} в случае если в качестве ключа используется закрытый класс в котором не определена своя реализация функции \path{hashCode}.
\end{itemize}

\subsection{Направления дальнейших работ}

\begin{itemize}
\item Пересмотреть реализации контейнеров в стандартной библиотеке.
\item Продолжить оптимизацию генерируемого кода.
\item Написать новые бенчмарки. 
\item Сделать чтобы генерируемый код был более идеомотичным для JavaScript (улучшить структуру генерируемого кода).
\item Добавить поддержку технологии source-map для удобной отладки сгенерированного кода.
\item Реализовать отсутствующие JavaScript компиляторе языковые фичи.
\item Оптимизации на основе Continuation-passing style и 
Static single assignment
\end{itemize}