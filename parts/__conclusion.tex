\startconclusionpage
%\section { Заключаение}

% Квалификационной работа завершается заключительной частью, которая так и называется «заключение». В заключении должно быть дано последовательное, логически стройное изложение полученных результатов и указано их соотношение с общей целью и конкретными задачами, сформулированными во введении. Именно здесь содержится, так называемое, «выводное» знание, которое является новым по отношению к исходному знанию и выносится на обсуждение и оценку в процессе защиты работы.
% Заключительная часть предполагает также наличие обобщенной итоговой оценки проделанной работы с точки зрения ее новизны, теоретической и практической значимости. При этом важно указать, в чем заключается главный смысл работы, какие важные побочные результаты получены, какие встают новые задачи в связи с проведением исследования. 
% Заключительная часть, составленная по такому плану, дополняет характеристику уровня профессиональной зрелости и квалификации автора работы.