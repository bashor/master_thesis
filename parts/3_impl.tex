\chapter{Реализация}

С кем сравниваемся?

JavaScript (бенчмарки от разработчиков v8)
Dart (бенчмарки от разработчиков)
Dart2js


Есть набор тестов которые используются практически всеми разработчиками JS движков. И было бы логично взять эти тесты и для моей работы, потому что можно взять использовать уже готовые, написанные специалистами тесты, тем самым исключив предвзятость с моей стороны. И конечно же я тоже опубликую свои тесты и каждый сможет сам провести такие же эксперименты как и я.

Отдельно хочется отметить последний тест -- этот бенчмарк пока еще не является общепринятым, но все же имеется реализация для многих языков. Данный дест был разработан разработчиками их гугла в 2011 с целью сравнения C++, Java, Scala, Go по таким критериями как производительность, объем кода, обхем бинарника, объем используемой памяти. 
Данный бенчмарк я также переписал на котлин, но пока не исследовал.

Тесты отмеченные одной звездочкой не реализованы.

Richards
OS kernel simulation benchmark, originally written in BCPL by Martin Richards (539 lines).
Main focus: property load/store, function/method calls
Secondary focus: code optimization, elimination of redundant code

Deltablue
One-way constraint solver, originally written in Smalltalk by John Maloney and Mario Wolczko (880 lines).
Main focus: polymorphism
Secondary focus: OO-style programming
---
Сравниваться со всем кончено получилось бы и пришлось выбрать нескольких.
Почему именно эти аналоги?
Ну во первых это реализация на чистом JS, были взяты тесты от разработчиков v8. Я считаю это обязательным пунктом, чтобы показать, понять на сколько мы хуже чем код написанный вручную на целевом языке.(тем более экспертами) И сделать выводы в чем мы хуже и почему, что можно улучшить. А в итого показать на сколько мы хуже или лучше чем такой код.

Казалось бы что может быть лучше вручную написанного JS кода? Я решил немного повысить планку и взять Dart, особенность в том, что дарт спроектирован с учетом некоторых недостатков JS, язык разрабоатывается практически той же коммандой которая разрабатывает v8, нативная реализация дарта выполняется в своей виртуальной машине которая знает о коде чуть больше и очень активно пользуется этим, за счет чего на многих тестах уже сейчас показывает лучшие результаты чем вручную написанный JS

Ну и планачка чуть по ниже это js полученный путем трансляции дарт в js

ТО план минимум работать так же хорошо как dart2js, план максимум работать так же хорошо как dart(видимо не всегда достижимый)
---
Richards
Данный бенчмар симулирует работу ядра ОС, написан в конце 90х Мартином Ричардсом.
Основной фокс теста -- это чтение и запись данных и вызов функций- методов

OS kernel simulation benchmark, originally written in BCPL by Martin Richards (539 lines).
Main focus: property load/store, function/method calls
Secondary focus: code optimization, elimination of redundant code

---
Richards: Выводы

Оптимизировать доступ к полям/свойствам, по возможности, сохранив бинарную совместимость
Сделать аналог импорта, предоставив тем самым возможность кратко обращаться к объектам и их полям
Необходимо упростить структуру генерируемого кода

---
Deltablue
constraint solver написанный John Maloney and Mario Wolczko в конце 80х, начале 90х (?)
Основной фокус данного теста -- полиморфизм, ООП стиль программирования. В тесте очень часто и в большем количестве создаются объекты. Эти объекты живут не очень долго и видимо этот тест так же подходит для тестирования GC, но это нас не сильно интересует

One-way constraint solver, originally written in Smalltalk by John Maloney and Mario Wolczko (880 lines).
Main focus: polymorphism
Secondary focus: OO-style programming

---
<<Прямой>> вызов конструкторов
---
DeltaBlue: Выводы

Заменить все контейнеры на родные для JS аналоги
Генерировать чистое прототипное наследование без оберток
Инициализировать все поля класса в конструкторе
Инициализировать все поля класса в одном порядке
Заменить или оптимизировать Kotllin.equals

---
Что дальше?

Реализовать предложенные оптимизации
Написать новые бенчмарки

Если новые бенчмарки не дадут результатов: (или если будет время)
Inlining функций и функциональных литералов
Оптимизации на основе  Continuation-passing style и 
Static single assignment
