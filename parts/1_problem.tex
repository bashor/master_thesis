\chapter{Постановка задачи}

Цель
Улучшить компилятор Kotlin в JavaScript так, чтобы он генерировал более эффективный в плане производительности код.


Задачи
Для достижения данной цели были поставлены следующие задачи
Изучить аналоги
Изучить существующие JS виртуальные машины (они давно перестали быть интерпритаторами)
Выбрать и реализовать тесты производительности для сравнения (бенчмарки)
Проанализировать результаты тестов и выделить <<узкие места>>
Предложить способы их оптимизации
Реализовать предложенные решения

------
Небольшое введение:
JavaScript — прототипно-ориентированный скриптовый язык программирования с динамической типизацией. Является диалектом языка ECMAScript.
Kotlin — статически типизированный объектно-ориентированный язык, разрабатываемый в компании JetBrains. Компилируется в Java байт-код и в JavaScript.

---

JavaScript is the x86 of the web
Brendan Eich (создатель JavaScript)

...JavaScript is the VM of the web. ....
Douglas Crockford (известный JS-разработчик, создатель JSON)

JavaScript is an assembly language for the Web. ...
Erik Meijer (language designer из Microsoft,
<<отец>> LINQ )

Почему JS? JS за последнее время стал очень популярен и его часто называют ассемблером для веба. Сейчас наверно почти для любого языка есть какой нибудь транслятор/компилятор в JS. 

---
Почему JavaScript?

Интерпретатор JS есть почти везде
JS не очень удобен для разработки больших проектов
Удобно писать все части системы на одном языке



Почему так происходит и почему команде котолина нужно изобретать свой ввелосипед? Я вижу несколько основных причин, думаю, общих для всех подобных проектов:
1. JS есть почти везде (потому что там есть браузер)
2. Но JS не очень удобен для разработки больших и сложных проектов (да и вообще для разработки)
3. Удобно писать и серверную и клиентскую часть(если такие есть) на одном языке. Да можно писать серверную часть на JS, но смотри 2
Дальше мнения могут расходится, например кому то нравится статическая типизация, кому то динамическая у каждого подхода есть свои минусы и плюсы...
или кому то нравятся много скобочек
Kotlin новый языка который тоже нуждается в таком трансляторе.
Ввиду специфики языка не так просто взять и переиспользовать готовые решения. Готовые решения обычно заточены под свой конкретный язык
Сейчас уже есть какая то реализация данного транслятора, но он еще достаточно далек от того что хотелось бы иметь.

---




JavaScript -- это язык програмимрования для Веб. Подовляющее большинство веб-сайтов используют JavaScript, и все современные веб-браузеры -- для настольных компьютеров, игровые приствок, эклектронных планшетов и смартфонов включают интерпритатор JavaScript, что делает JavaScript самым широкоприменимым языком программирования из когда либо существовавших в истории. JavaScript входит в тройку техналогий, которые должны знать  любой веб-разработчик: язык разметки HTML, позволяющий определить содержимов веб-страниц, язык стилей CSS, позволяющий определить внешний вид веб-страниц, и язык программирования JavaScript, позволяющий определять поведение веб-страниц.

JavaScript является высокоуровневым, динамическим, нетипизрованым и интрепретируемым языком программирования, который хорошо подходит для программирования в объектно-ориентированном и функциональном стилях. Свой синтаксис JavaScript унаследовал из языка Java, свои первоклассные функции -- из языка Scheme, а механизм наследования на основе прототипов – из языка Self. 

Название языка «JavaScript» может вводить в заблуждение. За иснлюкечинем поверхностной синтаксической схожести, JavaScript полностью отличается от языка программирования Java. JavaScript давно перерос рамки языка сценариев, превратившись в надежный и эффективный универсальный язык программирования. Последняя версия языка (смотрите врезку) определяет множество новых особенностей, позволяющих использовать его для разработки крупномасштабного программного обеспечения.


Чтобы представлять хоть какой-то интерес, каждый язык программирования должен иметь свою платформу, или стандартную библиотеку или API функций для выполнения таких базовых операций, как ввод и  вывод. Ядро языка JavaScript определяет минимальный прикладной интерфейс для работы с текстом, массивами, датами и регулярными выражениями, но в нем отсутствуют операции ввода-вывода. Ввод и вывод (а также более сложные возможности, такие как сетевые взаимодействия, сохранение данных и работа с графикой) перекладываются на «окружающую среду», куда встраивается JavaScript. Обычно роль окружающей среды играет веб-броузер (однако в главе 12 мы увидим два примера использования JavaScript без привлечения веб-броузера).

JavaScript – объектно-ориентированный язык, но используемая в нем объектная модель в корне отличается от модели, используемой в большинстве других языков. 


---

JavaScript: названия и версии
JavaScript был создан в компании Netscape на заре зарождения Веб. Название «JavaScript» является торговой маркой, зарегистрированной компанией Sun Microsystems (ныне Oracle), и  используется для обозначения реализации языка, созданной компанией Netscape (ныне Mozilla). Компания Netscape представила язык для стандартизации европейской ассоциации производителей компьютеров ECMA (European Computer Manufacturer’s Association), но из-за юридических проблем с торговыми марками стандартизованная версия языка получила несколько неуклюжее название «ECMAScript». Из-за тех же юридических проблем версия языка от компании Microsoft получила официальное название «JScript». Однако на практике все эти реализации обычно называют JavaScript. В течение прошлого десятилетия все веб-броузеры предоставляли реализацию версии 3 стандарта ECMAScript, и в действительности разработчикам не было необходимости задумываться о номерах версий: стандарт языка был стабилен, а его реализации в веб-броузерах в значительной мере были совместимыми. Недавно(???) вышла новая важная версия стандарта языка под названием ECMAScript 5, и к моменту написания этих строк производители броузеров приступили к созданию его реализации. Эта книга охватывает все нововведения, появившиеся в ECMAScript 5, а также все особенно
сти, предусмотренные стандартом ECMAScript 3. (Четвертая версия стандарта ECMAScript разрабатывалась много лет, но из-за слишком амбициозных целей так и не была выпущена.) Однако иногда можно встретить упоминание о версии JavaScript, например: JavaScript 1.5 или JavaScript 1.8. Эти номера версий присваивались реализациям JavaScript, выпускаемым компанией Mozilla, причем версия 1.5 соответствует базовому стандарту ECMAScript 3, а более высокие версии включают нестандартные расшире
ния. Наконец, номера версий также присваиваются отдельным интерпретаторам, или «механизмам» JavaScript. Например, компания Google разрабатывает свой интерпретатор JavaScript под названием V8, и к моменту написания этих строк текущей версией механизма V8 была версия 3.0.\cite{JsDef6}
